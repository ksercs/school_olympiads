\begin{problem}{Долгое путешествие}{knumbers.in}{knumbers.out}{2 секунды}{256 мегабайт}
                                                                    
%Автор задачи: Дмитрий Филиппов
%Автор условия: Дмитрий Филиппов

Мэтр очень любит путешествовать и смотреть на мир вокруг. Но еще больше он любит смотреть на изменение своего счетчика пробега.
Недавно он заметил одно интересное свойство у последних чисел, которые он видел на счетчике.
Оказалось, что в каждом из них количество различных цифр не превосходило $k$. Мэтр решил, что это все неспроста, и
назвал такие числа <<замечательными>>. После этого он посмотрел на текущее показание
счетчика и задался вопросом: а когда в следующий раз на нем будет показано замечательное число? Понимая, что эта задача
ему не под силу, Мэтр обратился к вам за помощью.

\InputFile
В первой строке входного файла дано число $k$ ($1 \le k \le 10$)~--- ограничение на количество различных цифр в замечательных числах.

Во второй строке дано число $x$ ($1 \le x \le 10^{10^6}$)~--- последнее показание счетчика, которое увидел Мэтр. Гарантируется, что в числе $x$,
нет ведущих нулей.
                                                 
\OutputFile
В единственной строке выходного файла выведите единственное число $y$ ($y \ge x$)~--- показание счетчика, такое, что в
числе $y$ не более $k$ различных цифр и ($y-x$)~--- минимально. Ответ не должен содержать ведущих нулей.

\Example
\begin{example}%
\exmp{
1
4321
}{
4444
}%
\exmp{
2
1234
}{
1311
}%
\exmp{
3
123
}{
123
}%
\end{example}

\end{problem}