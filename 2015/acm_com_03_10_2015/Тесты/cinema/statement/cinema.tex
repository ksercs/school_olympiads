\begin{problem}{Поход в кино}{cinema.in}{cinema.out}{2 секунды}{256 мегабайт}
                                                                    
%Автор задачи: Дмитрий Филиппов
%Автор условия: Дмитрий Филиппов

Сегодня премьера нового фильма <<Тазики>>, и Молния Маккуин обязательно хочет на него сходить. Однако, одному идти скучно, поэтому он
решил взять с собой друзей. Все бы хорошо, но Маккуин знает, что на премьеру обязательно пойдет его давний враг Чико Хикс, с которым он
и его друзья совсем не хотят видеться, а тем более сидеть рядом.

Маккуин хорошо знает Чико, поэтому уверен, что тот сядет в центр зала. Так же он знает, что в зале будет ровно $N$ рядов по $N$ мест.
Хорошенько все обсудив с друзьями, Маккуин решил, что они все-таки пойдут на премьеру. Но с одним условием. Он и его друзья будут сидеть
хотя бы в $A$ рядах и хотя бы в $B$ местах от Чико. Это означает, что если Маккуин или его друг сидит на ряду номер $x_1$, на месте номер $y_1$,
а Чико сидит на ряду номер $x_2$, месте номер $y_2$, то должны выполняться два условия: $|x_1-x_2| \ge A$, $|y_1-y_2| \ge B$.

Теперь Маккуин хочет понять, какое максимальное количество друзей он может взять с собой.

\InputFile
В первой и единственной строке входного файла дано три числа $N, A, B$ ($1 \le N, A, B \le 1000$, $N$~--- нечетное)~--- количество рядов
и мест в зале, а также ограничения из условия.

\OutputFile
В единственной строке выходного файла выведите количество друзей, которое Маккуин сможет взять с собой (не считая его самого).
Если подходящего места нет даже для Маккуина, в единственной строке выходного файла выведите 0.

\Example
\begin{example}%
\exmp{
3 1 1
}{
3
}%
\exmp{
3 1 2
}{
0
}%
\end{example}

\end{problem}