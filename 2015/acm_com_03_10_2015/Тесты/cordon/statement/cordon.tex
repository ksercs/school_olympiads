\begin{problem}{Оцепление}{cordon.in}{cordon.out}{2 секунды}{256 мегабайт}
                                                                    
%Автор задачи: Илья Пересадин
%Автор условия: Илья Пересадин

Из тюрьмы сбежала особо опасная тачка~--- Маттэо и уже мчится по Шоссе 66 к городку Радиатор-Спрингс. 
Полицейский Шериф хочет поймать Маттэо, когда тот будет проезжать по городу. 

Радиатор-Спрингс разбит на $n\times m$ кварталов, то есть план города представляет собой матрицу $n\times m$.
Квартал с координатами $(1, 1)$, который находится в левом верхнем углу ~--- это въезд в город, 
квартал с координатами $(n, m)$, который находится в правом нижнем углу ~--- выезд из города.

Маттэо въедет в город и начнет свое движение к выезду. Он очень спешит, поэтому будет двигаться только  
вправо или вниз, то есть находясь в квартале с координатами $(i, j)$, он поедет в квартал $(i+1, j)$ или $(i, j+1)$.
Но Шериф пока не знает, какой конкретно маршрут выберет Маттэо, поэтому Шериф хочет оцепить некоторые кварталы.
Он хочет сделать это так, чтобы какой бы маршрут ни выбрал Маттэо, он обязательно проедет как минимум по $k$ оцепленным кварталам. 
При этом Шериф хочет минимизировать количество оцепленных кварталов, чтобы как можно меньше будоражить город. 
Так же он не собирается оцеплять въезд и выезд, чтобы не затруднять движение других тачек. 

Помогите Шерифу поймать Маттэо.

\InputFile
Единственная строка входного файла содержит три натуральных числа $n$, $m$, $k$ ($1 \le n, m \le 300$, $0 \le k \le 10^9$, $n\times m > 1$)~--- 
размер города и минимальное количество оцепленных кварталов, которое должно встретиться Маттэо на пути от въезда до выезда.

\OutputFile
В первой строке выведите <<YES>>, если решение существует или <<NO>> в противном случае.
В случае положительного ответа в следующих строках выведите матрицу $n\times m$~--- план оцепления, каждый символ которой либо 'C', обозначающий оцепленный квартал, либо
'.'~--- свободный квартал.

\Example
\begin{example}%
\exmp{
2 2 1
}{
YES
.C
C.
}%
\exmp{
1 6 2
}{
YES
..CC..
}%
\exmp{
3 3 4
}{
NO
}%
\end{example}

\end{problem}
