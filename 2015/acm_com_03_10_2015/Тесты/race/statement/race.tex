\begin{problem}{Гонка}{race.in}{race.out}{2 секунды}{256 мегабайт}
                                                                    
%Автор задачи: Илья Пересадин
%Автор условия: Илья Пересадин

В Радиатор-Спрингс назревает гонка. Каждая уважающая себя тачка в городе хотела бы поучаствовать в ней.
На данный момент на нее записались $n$ тачек. У каждой тачки есть автомобильный номер, в Радиатор-Спрингс это непустая строка, 
состоящая из строчных латинских букв. Номера двух тачек необязательно различны.

Но участие в гонке смогут принять не все. Судьи мероприятия сами выбирают машины, которые примут в ней участие.
Они хотят максимизировать \textit{зрелищность} гонки. \textit{Зрелищность} гонки~--- это натуральное число, 
которое равно произведению трех величин: 
\begin{itemize}
  \item количество машин, участвующих в гонке;
  \item длина наибольшего общего префикса номеров машин, участвующих в гонке;
  \item длина наибольшего общего суффикса номеров машин, участвующих в гонке.
\end{itemize}

Наибольшим общим префиксом множества строк $s_1$, $s_2$ \dots $s_n$ называется наибольшая по длине строка, которая является префиксом каждой строки $s_i$.

Наибольшим общим суффиксом множества строк $s_1$, $s_2$ \dots $s_n$ называется наибольшая по длине строка, которая является суффиксом каждой строки $s_i$.

Такое определение \textit{зрелищности} связано с фотографиями, которые делаются во время гонок. Чем больше <<похожи>> номера мчащихся рядом тачек, 
тем больше эстетического удовольствия доставляют фотографии.

Помогите судьям выбрать подмножество машин с наибольшей \textit{зрелищностью}.

\InputFile
В первой строке входного файла содержится одно натуральное число $n$ ($1 \le n \le 2\cdot 10^5$) ~--- количество машин, желающих принять участие в соревновании.
В следующих $n$ строках содержатся $n$ непустых строк, состоящих из строчных латинских букв.

Суммарная длина строк не превосходит $2\cdot 10^5$.

\OutputFile
Единственная строка выходного файла должна содержать целое число~--- наибольшую \textit{зрелищность} гонки, которую можно достичь.

\Example
\begin{example}%
\exmp{
4
aa
aa
aa
bb
}{
12
}%
\exmp{
4
aba
ab
abbba
aba
}{
25
}%
\exmp{
8
abcccbx
abccbx
abcycbx
abczcbx
abcbx
abcacbx
abcscbx
axcccbx
}{
63
}%
\end{example}

\Note
В первом примере в гонке будут участвовать первые три машины. Длина их наибольшего общего префикса~--- 2, суффикса~--- 2, количество~--- 3, получаем $3 \times 2 \times 2 = 12$.

Во втором примере в гонке будет участвовать только третий автомобиль. Наибольший общий префикс и суффикс одной строки совпадает с этой строкой, поэтому получаем $1\times 5 \times 5 = 25$.

В третьем примере в гонке будут участвовать все автомобили, кроме последнего. Длина их наибольшего общего префикса~--- 3, суффикса~--- 3, всего автомобилей~--- 7, получаем $7 \times 3 \times 3 = 63$.
\end{problem}