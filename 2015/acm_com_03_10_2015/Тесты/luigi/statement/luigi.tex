\begin{problem}{Расчеты Луиджи}{luigi.in}{luigi.out}{2 секунды}{256 мегабайт}
                                                                    
%Автор задачи: Илья Пересадин
%Автор условия: Григорий Шовкопляс

Луиджи знает толк в подборе шин. Недавно он попросил Гвидо проводить расчеты для определения оптимального сочетания трех факторов: дорога, шины и диски.

В расчетах Луиджи дорога характеризуется числом $k$~--- частотой микровыбоин, а шины и диски числами $x$ и $y$~--- коэффициентами крутости
по личной шкале Луиджи. Гвидо опытно установил, что обязательным условием сочетания факторов является выполнение формулы:
$x + y \equiv 0 \mod k$. То есть сумма коэффициентов крутости шин и дисков должна нацело делиться на частоту микровыбоин.

Луиджи всегда работает на максимуме возможностей и хочет подбирать в своем магазине самые крутые шины, учитывая ассортимент.
В магазине Луиджи можно купить шины с любыми коэффициентами крутости от $A$ до $B$, аналогично диски коэффициентами от $C$ до $D$.
При этом если вариантов, подходящих под данную дорогу несколько, то Луиджи подбирает любой из тех, в котором сумма коэффициентов максимальна.


Луиджи просит вас, написать программу, которая по числам $A$, $B$, $C$, $D$ и $k$ определит какие шины и какие диски нужно продать.    

\InputFile
В первой и единственной строке входного файла дано пять натуральных чисел $A, B, C, D, k$ ($1 \le A, B, C, D, k \le 10^9$)~--- ограничения
на ассортимент, а также частота микровыбоин дороги.

\OutputFile
В единственной строке выходного файла выведите два числа $x$ и $y$ ($A \le x \le B, C \le y \le D$)~--- ответ на задачу. 
Если ответа не существует, выведите \texttt{-1}.

\Example
\begin{example}%
\exmp{
1 10 1 7 3
}{
9 6
}%
\exmp{
1 2 2 3 7
}{
-1
}%
\exmp{
1 4 2 3 2
}{
4 2
}%
\end{example}

\end{problem}