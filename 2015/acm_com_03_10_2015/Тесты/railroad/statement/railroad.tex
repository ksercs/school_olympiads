\begin{problem}{Железная дорога}{railroad.in}{railroad.out}{2 секунды}{256 мегабайт}
                                                                    
%Автор задачи: Захар Войт
%Автор условия: Илья Пересадин

Через Радиатор-Спрингс проходит крупнейшая железная дорога. 
Введем на ней систему координат.
По железной дороге движутся $n$ поездов. Каждый поезд представляет собой отрезок. 
Поезд с номером $i$ в начальный момент времени занимает отрезок $[a_i;b_i]$. Поезда не стоят на месте~--- $i$-й поезд движется с постоянной скоростью $v_i$. 
Железная дорога двунаправлена, то есть поезда могут двигаться как в положительном направлении оси, так и в отрицательном.
Отрезки, представляющие поезда, в любой момент времени могут пересекаться, вкладываться друг в друга и совпадать.

В точке $x$ находится железнодорожный переезд, к которому в моменты $t_i$ подъезжают тачки.
Для каждой тачки требуется вычислить минимальный момент времени, в который она сможет пересечь железнодорожный переезд.

Тачка может пересечь переезд, если он не занят поездом. Переезд считается занятым, если отрезок, представляюший
собой некоторый поезд, содержит в себе точку $x$. 
Причем если поезд подъезжает к перекрестку одновременно тачкой, то переезд считается занятым.
Тачки пересекают переезд мгновенно.

\InputFile
В первой строке входного файла три целых числа~--- $n$, $m$ и $x$ ($1 \le n, m \le 10^5, |x| \le 10^9$) количество поездов, движущихся по железной дороге, количество тачек, 
подъезжающих к железнодорожному переезду и точка, в которой находится переезд.

В следующих $n$ строках содержатся по три целых числа $a_i, b_i, v_i$ ($|a_i| \le 10^9$, $|b_i| \le 10^9$, $1 \le v_i \le 10^9, a_i \ne b_i$)~--- 
отрезок, задающий поезд и его скорость движения. Если $a_i < b_i$, то поезд движется в положительном направлении оси, если $a_i > b_i$~--- в отрицательном.

В следующей строке находятся $m$ неотрицательных целых чисел $t_j$ ($0 \le t_j \le 10^9$)~--- моменты времени, в которые к переезду подъедут тачки.

\OutputFile
В $m$ строках выходного файла выведите $m$ вещественных чисел $b_j$ ~--- минимальный момент времени, в который $j$-я тачка сможет пересечь железнодорожный переезд.
Ответ будет считаться правильным, если относительная или абсолютная погрешность каждого $b_j$ не превосходит $10^{-6}$.

\Example
\begin{example}%
\exmp{
3 2 0
-4 -1 1
13 6 3
-7 -6 1
1 5
}{
4.333333333
5.000000000
}%
\exmp{
2 2 0
4 2 1
-11 -8 2
2 6
}{
5.500000000
6.000000000
}%
\end{example}

\end{problem}
