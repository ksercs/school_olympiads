\begin{problem}{Новобранцы}{soldiers.in}{soldiers.out}{2 секунды}{256 мегабайт}
                                                                    
%Автор задачи: Евгений Замятин
%Автор условия: Евгений Замятин
По заказу правительства США был разработан новый инновационный вид тачек. Для того, чтобы натренировать и испытать их в действии, эти тачки были направлены
главной военной тачке страны~--- Сержанту. Узнав об этом, Сержант очень обрадовался. Однако очень быстро от его радости не осталось и следа. Больше всего Сержанта 
расстроила недисциплинированность новобранцев. Им было принято решение это исправить.

Сержант построил $n$ тачек в шеренгу. Из них $m$ автомобилей являются новобранцами. Соответственно, $(n - m)$ машин <<старой закалки>>.
Сержант отдал $k$ команд, каждая из которых имеет вид:
\begin{itemize}
    \item <<Налево!>>~--- поворот на $90^{\circ}$ налево;
    \item <<Направо!>>~--- поворот на $90^{\circ}$ направо;
    \item <<Кругом!>>~--- поворот на $180^{\circ}$.
\end{itemize} 
Известно, что старые машины выполняют все команды правильно, а новые всё делают наоборот. Иными словами, когда новая машина слышит команду <<Налево!>>, она поворачивает
направо, когда слышит команду <<Направо!>>, она поворачивает налево. Команду <<Кругом>> новые машины выполняют правильно. 
Этот беспредел Сержанту очень не понравился. После каждой команды он смотрел на всю шеренгу, и если в ней было две машины, направленные в разные стороны,
он заставлял всю шеренгу как следует хлопнуть дверьми.
Сержанту стало интересно, сколько раз ему пришлось наказать своих подчинённых. 
Помогите ему посчитать это количество, чтобы бедным машинам не пришлось снова терпеть наказания!

\InputFile
В первой строке входных данных даны числа $n$, $m$ ($1 \le n \le 10^5, 0 \le m \le n$)~--- общее количество машин в шеренге и количество новобранцев.
Во второй строке дано число $k$ ($1 \le k \le 10^5$)~--- число команд, отданных Сержантом.
В третьей строке дано описание команд~--- строка длиной ровно $k$ символов, каждый символ в которой обозначает команду: <<L>>~--- налево, <<R>>~--- направо, <<A>>~--- кругом.
Гарантируется, что строка состоит только из данных символов.

\OutputFile
Выведите количество раз, которое Сержанту пришлось наказать своих подчинённых.

\Example
\begin{example}%
\exmp{
5 3
7
LRARLRL
}{
3
}%
\end{example}

\end{problem}