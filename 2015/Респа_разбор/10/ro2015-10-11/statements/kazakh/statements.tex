\documentclass [11pt, a4paper, oneside] {article}

\usepackage [T2A] {fontenc}
\usepackage [utf8] {inputenc}
\usepackage [english, russian] {babel}
\usepackage {amsmath}
\usepackage {amssymb}
\usepackage [kazakh]{olymp}
\usepackage {comment}
\usepackage {epigraph}
\usepackage {expdlist}
\usepackage {graphicx}

\begin {document}

\contest
{Информатика пәні бойынша Республикалық олимпиада 2015-нің 4-шi кезеңі, 10-11 сынып}%
{Орал}%
{13-18 наурыз, 2015}%

\binoppenalty=10000
\relpenalty=10000

\renewcommand{\t}{\texttt}

\graphicspath{{../../problems/array/statements/kazakh/}}
\input ../../problems/array/statements/kazakh/problem.tex
\graphicspath{{../../problems/cafe/statements/kazakh/}}
\input ../../problems/cafe/statements/kazakh/problem.tex
\graphicspath{{../../problems/pair-of-nonintersected-intervals/statements/kazakh/}}
\input ../../problems/pair-of-nonintersected-intervals/statements/kazakh/problem.tex
\graphicspath{{../../problems/lcmer/statements/kazakh/}}
\input ../../problems/lcmer/statements/kazakh/problem.tex
\graphicspath{{../../problems/2n-matching/statements/kazakh/}}
\input ../../problems/2n-matching/statements/kazakh/problem.tex
\graphicspath{{../../problems/compatible/statements/kazakh/}}
\input ../../problems/compatible/statements/kazakh/problem.tex

\end {document}
