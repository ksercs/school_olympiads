\begin{problem}{Үйлесімді сандардың саны}{F.in}{F.out}{0.5 seconds}{256 megabytes}

Егер екі X және Y бүтін сандарының бит бойынша <<Және>> операциясының нәтижесі нөлге тең, яғни X AND Y = 0 болса, онда оларды үйлесімді деп айтамыз. Мысалы,  77 (1001101${}_{2}$) мен 50 (110010${}_{2}$) үйлесімді, өйткені олардың екілік жүйедегі <<Және>> операциясының нәтижесі нөлге тең 1001101${}_{2}$ AND 110010${}_{2}$=0${}_{2}$, ал  3 (11${}_{2}$) пен 6 (110${}_{2}$) үйлесімді емес, себебі олардың екілік жүйедегі <<Және>> операциясының нәтижесі 11${}_{2}$ AND 110${}_{2}$=10${}_{2}$ нөлге тең емес.


Сізге бүтін сандардан тұратын A${}_{1}$, A${}_{2}$,...,A${}_{N}$ массиві берілген. Массивтің әрбір элементіне онымен осы массивтегі үйлесімді сандардың санын табу керек.


\InputFile
Енгiзу файлының бiрiншi жолында бiр оң бүтiн сан $N$ ($1 \le N \le 10^5$) --- массивтегі элементтердің саны берілген. Екінші жолда бос орын арқылы $N$ бүтін сандар берілген A${}_{1}$, A${}_{2}$, ..., A${}_{N}$ ($1 \le A_i \le 4 \cdot 10^6$) --- берілген массивтің элементтері. Массивте сандар қайталануы мүмкін.

\OutputFile
Шығару файлына N бүтін санды бос орын арқылы шығарыңыз, яғни массивтің әрбір i-шы элементіне осы массивтегі онымен үйлесімді сандардың саны.

\Examples

\begin{example}
\exmp{2
50 77
}{1 1 }%
\exmp{5
1 2 3 4 5
}{2 3 1 3 1 }%
\exmp{7
2 7 8 2 6 10 1
}{2 1 5 2 2 1 5 }%
\end{example}

\Note
Бірінші мысалда A${}_{1}$ элементі A${}_{2}$ элементімен үйлесімді, сондықтан жауабы: 1 1. 

Екінші мысалда A${}_{1}$ элементі A${}_{2}$, A${}_{4}$ элементтерімен үйлесімді, A${}_{2}$ элементі A${}_{1}$, A${}_{4}$, A${}_{5}$ элементтерімен үйлесімді, A${}_{3}$ элементі A${}_{4}$ элементімен үйлесімді, A${}_{4}$ элементі A${}_{1}$, A${}_{2}$, A${}_{3}$ элементтерімен үйлесімді, A${}_{5}$ элементі A${}_{2}$ элементімен үйлесімді, сондықтан жауабы: 2 3 1 3 1.

C/С++ тілінде бит бойынша <<Және>> операциясын \& операторымен орындауға болады.

Pascal тілінде бит бойынша <<Және>> операциясын and операторымен орындауға болады.

\Scoring
Берілген тапсырма үш бөліктен тұрады:
\begin{enumerate}
\item ($1 \le A_i \le 4 \cdot 10^6$) және $1 \le N \le 10^4$. Бағалануы $25$ ұпай.
\item ($1 \le A_i \le 10^2$) және $1 \le N \le 10^5$. Бағалануы $25$ ұпай.
\item ($1 \le A_i \le 4 \cdot 10^6$) және $1 \le N \le 10^5$. Бағалануы $50$ ұпай.
\end{enumerate}

\end{problem}
