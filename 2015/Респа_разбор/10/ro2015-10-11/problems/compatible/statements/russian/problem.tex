\begin{problem}{Количество совместимых чисел}{F.in}{F.out}{0.5 секунд}{256 мегабайт}

Два целых числа X и Y называются совместимыми, если результат их побитового «И» равен нулю, то есть X AND Y = 0. Например, числа 77 (1001101${}_{2}$) и 50 (110010${}_{2}$) совместимы, так как 1001101${}_{2}$ AND 110010${}_{2}$=0${}_{2}$, а числа 3 (11${}_{2}$) и 6 (110${}_{2}$) несовместимы, так как 11${}_{2}$ AND 110${}_{2}$=10${}_{2}$.

Вам дан массив целых чисел A${}_{1}$, A${}_{2}$,..., A${}_{N}$. Требуется определить для каждого элемента массива, количество совместимых элементов с ним в данном массиве. 

\InputFile
В первой строке записано целое число $N$ ($1 \le N \le 10^5$) --- количество элементов в данном массиве. Во второй строке через пробел записаны $N$ целых чисел A${}_{1}$, A${}_{2}$, ..., A${}_{N}$ ($1 \le A_i \le 4 \cdot 10^6$) --- элементы данного массива. Числа в массиве могут повторяться.

\OutputFile
Выведите N целых чисел через пробел, то есть количество совместимых чисел для каждого i-го элемента массива.

\Examples

\begin{example}
\exmp{2
50 77
}{1 1 }%
\exmp{5
1 2 3 4 5
}{2 3 1 3 1 }%
\exmp{7
2 7 8 2 6 10 1
}{2 1 5 2 2 1 5 }%
\end{example}

\Note
В первом примере элемент A${}_{1}$ совместим с элементом A${}_{2}$, поэтому ответ: 1 1.

Во втором примере элемент A${}_{1}$ совместим с элементами A${}_{2}$, A${}_{4}$, элемент A${}_{2}$ совместим с элементами A${}_{1}$, A${}_{4}$, A${}_{5}$, элемент A${}_{3}$ совместим с элементом A${}_{4}$, элемент A${}_{4}$ совместим с элементами A${}_{1}$, A${}_{2}$, A${}_{3}$, элемент A${}_{5}$ совместим с элементом A${}_{2}$, поэтому ответ:
1 2 3 4 5.

В C/С++ операция побитового «И» реализовано при помощи оператора \&.

В Pascal операция побитового «И» реализовано при помощи оператора and.

\Scoring

Данная задача содержит три подзадачи:
\begin{enumerate}
\item ($1 \le A_i \le 4 \cdot 10^6$) и $1 \le N \le 10^4$. Оценивается в $25$ баллов.
\item ($1 \le A_i \le 10^2$) и $1 \le N \le 10^5$. Оценивается в $25$ баллов.
\item ($1 \le A_i \le 4 \cdot 10^6$) и $1 \le N \le 10^5$. Оценивается в $50$ балл.
\end{enumerate}

\end{problem}
