\begin{problem}{Кафе}{B.in}{B.out}{0.5 секунд}{256 мегабайт}

Бүгін Нағыз Университетіндегі (НУ) кафеге $N$ студент келді. Әрбір келген студент бір шынаяқ кофе ішіп және бір бәліш жеп кеткісі келеді (ешбір студент тек кофе ішуге немесе тек бәліш жеуге келіспейді, бұл жағдайда студент кетіп қалады). Кафеде $M$ түрлі кофе және $K$ түрлі бәліш ұсынылған. Кофенің немесе бәліштің әрбір түрінен қанша бар екендігі де көрсетілген. 

Соған қарамастан әрбір студенттің өзіндік қалауы бар. Әрбір студент үшін кофенің және бәліштің қандай түрлерін ұнататыны белгілі. Ешбір студент өзі ұнатпайтын нәрсені ішіп-жеуге келіспейді.

Кафенің бастығы келген қанша студентке қызмет көрсете алатынын білгісі келеді. Ал сіз бұл санды санай аласыз ба? 


\InputFile
Бірінші қатар бүтін $N$, $M$, $K$, ($1 \le N, M, K \le 500$) сандарынан тұрады.

Екінші қатарда бос орындар арқылы бөлінген $M$ бүтін сан берілген $C_1,C_2,\dots,C_M$ ($1 \le C_i \le 500$) --- кофенің әрбір түрінің қанша шынаяғы бар екндігі.

Үшінші қатарда бос орындар арқылы бөлінген $K$ бүтін сан берілген $P_1,P_2,\dots,P_K$ ($1 \le P_i \le 500$) --- бәліштің әрбір түрінен қанша бар екендігі.

Келесі $N$ қатарда әрбір студенттің қандай кофе түрлерін ұнататыны жайлы мәлімет көрсетілген. $i$-ші қатар ($1 \le i \le N$) $X_i$ санынан және одан соң еретін әртүрлі $A_1,A_2,\dots,A_{X_i}$ сандарынан тұрады. Бұл $i$-ші студенттің ұнататын кофелерінің түрлері.

Келесі $N$ қатарда әрбір студенттің қандай бәліш түрлерін ұнататыны жайлы мәлімет көрсетілген. $i$-ші қатар ($1 \le i \le N$) $Y_i$ санынан және одан соң еретін әртүрлі $B_1,B_2,\dots,B_{Y_i}$ сандарынан тұрады. Бұл $i$-ші студенттің ұнататын бәліштерінің түрлері.

\OutputFile
Өзінің ұнайтын кофесін және бәлішін ішіп-жей алатын ең көп дегендегі студенттердің санын шығарыңыз.

\Examples

\begin{example}
\exmp{2 3 1
5 1 3
2
3 1 2 3
1 2
1 1
1 1
}{2
}%
\end{example}


\Scoring
Берілген тапсырма үш бөліктен тұрады:
\begin{enumerate}
\item $1 \le N, M, K \le 5$. Бүкіл $X_i$ мен $Y_i$ ($1 \le i \le N$) сандарының қосындысы $10$-нан аспайды. Бағалануы $21$ ұпай.
\item $1 \le N, M, K \le 20$. Бүкіл $X_i$ мен $Y_i$ ($1 \le i \le N$) сандарының қосындысы $15$-тен аспайды. Бағалануы $33$ ұпай.
\item $1 \le N, M, K \le 500$. Бүкіл $X_i$ мен $Y_i$ ($1 \le i \le N$) сандарының қосындысы $2000$-нан аспайды. Бағалануы $46$ ұпай.
\end{enumerate}

Әр бөлік өзінен алдынғы бөліктер орындалғанда ғана бағаланады.

\end{problem}
