\begin{problem}{Arti vs Mex-Mans}{D.in}{D.out}{0.5 секунд}{256 мегабайт}

Артемда $1 \le N \le 100$ саннан тұратын $x$ тізбегі болды. Бұл тізбектің келесі қасиеттері бар: $1 \le L \le x_1 \le x_2 \le \ldots \le x_N \le R \le 10^9$ және бұл сандардың ең кіші ортақ еселігі $1 \le A \le 10^9$ санына бөлінеді. Бірақ Мансұр келіп $x$ тізбегін ұрлап кетті. Енді Артемнің көңіл-күйі жоқ, өйткені оның өз тізбегіндегі сандар есінде жоқ. Ол тек $N$, $L$, $R$ және $A$ сандарын есінде сақтап қалды. Ол өз тізбегіндегі сандарды еске түсіргісі келеді. Одан бұрын ол $N$, $L$, $R$ және $A$ сандарына сәйкес келетін қанша тізбек бар екенін тапқысы келеді. Артемға көмектесіңіз, --- осы есепті шешетін бағдарлама жазыңыз.

\InputFile
Бірінші қатарда бос орындармен бөлінген төрт бүтін оң сан берілген: $N$, $L$, $R$, $A$.

\OutputFile
Тапсырманың жауабы болатын санды шығарыңыз. Жауап өте үлкен бола алғандықтан оны $10^9 + 7$ санына бөлгендегі қалдығын шығарыңыз.


\Examples

\begin{example}
\exmp{2 1 7 6
}{9
}%
\exmp{1 1 50 7
}{7
}%
\end{example}


Бірінші тестте келесі реттер керекті қасиеттерге сай келеді:

$\{ 1, 6\}$, $\{ 2, 3\}$, $\{ 2, 6\}$

$\{ 3, 4\}$, $\{ 3, 6\}$, $\{ 4, 6\}$

$\{ 5, 6\}$, $\{ 6, 6\}$, $\{ 6, 7\}$

\Scoring
Берілген тапсырма бес бөліктен тұрады:
\begin{enumerate}
\item $1 \le N \le 2$, $1 \le A,L,R \le 100$. Бағалануы $6$ ұпай.
\item $1 \le N \le 2$, $1 \le A,L,R \le 1000$. Бағалануы $11$ ұпай.
\item $1 \le N \le 10$, $1 \le A,L,R \le 1000$. Бағалануы $15$ ұпай.
\item $1 \le N \le 10$, $1 \le A,L,R \le 10^6$. Бағалануы $21$ ұпай.
\item $1 \le N \le 100$, $1 \le A,L,R \le 10^9$. Бағалануы $47$ ұпай.
\end{enumerate}

Әр бөлік өзінен алдынғы бөліктер орындалғанда ғана бағаланады.


\end{problem}
