\begin{problem}{Arti vs Mex-Mans}{D.in}{D.out}{0.5 секунд}{256 мегабайт}

У Артема была последовательность $x$ из $1 \le N \le 100$ чисел, для которой выполнялось следующее свойство $1 \le L \le x_1 \le x_2 \le \ldots \le x_N \le R \le 10^9$, а наименьшее общее кратное этих чисел делилось на $1 \le A \le 10^9$. Но пришел Мансур и украл последовательность $x$. Артем очень расстроился, ведь он не помнит значения чисел своей последовательности. Он помнит только числа $N$, $L$, $R$ и $A$. Он хочет восстановить последовательность. Для этого он решил сначала посчитать, а сколько вообще существует последовательностей, с такими же $N$, $L$, $R$ и $A$. Помогите ему, --- напишите программу для решения этой задачи.

\InputFile
Единственная строка входных данных содержит четыре целых положительных числа, разделенных  пробелами: $N$, $L$, $R$, $A$.

\OutputFile
Выведите единственное число, ответ на задачу. Так как ответ может быть очень большим, выведите его остаток от деления на $10^9 + 7$.

\Examples

\begin{example}
\exmp{2 1 7 6
}{9
}%
\exmp{1 1 50 7
}{7
}%
\end{example}


В первом тестовом примере подходящими последовательностями будут следующие:

$\{ 1, 6\}$, $\{ 2, 3\}$, $\{ 2, 6\}$

$\{ 3, 4\}$, $\{ 3, 6\}$, $\{ 4, 6\}$

$\{ 5, 6\}$, $\{ 6, 6\}$, $\{ 6, 7\}$

\Scoring
Данная задача содержит пять подзадач:
\begin{enumerate}
\item $1 \le N \le 2$, $1 \le A,L,R \le 100$. Оценивается в $6$ баллов.
\item $1 \le N \le 2$, $1 \le A,L,R \le 1000$. Оценивается в $11$ баллов.
\item $1 \le N \le 10$, $1 \le A,L,R \le 1000$. Оценивается в $15$ баллов.
\item $1 \le N \le 10$, $1 \le A,L,R \le 10^6$. Оценивается в $21$ балл.
\item $1 \le N \le 100$, $1 \le A,L,R \le 10^9$. Оценивается в $47$ баллов.
\end{enumerate}

Каждая следующая подзадача оценивается только при прохождении всех предыдущих.

\end{problem}
