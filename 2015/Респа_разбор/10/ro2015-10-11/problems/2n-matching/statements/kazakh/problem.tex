\begin{problem}{Наурыз Cup 2015}{E.in}{E.out}{0.5 секунд}{256 мегабайт}

Жуырда <<Наурыз Cup 2015>>  атты топтық сайысы болғалы тұр. Бір топта екі қатысушы болуы тиіс. Амантайдың бұл сайысқа қатты қатысқысы келеді. Ол сайысқа қатысқалы тұрған және өзін қоса есептегендегі барлық $2 \cdot N$ ($1 \le N \le 10^5$) қатысушының тізіміне қол жеткізді. Тізімде әрбір қатысушының рейтингі көрсетілген. Топтың рейтингі екі қатысушының орташа рейтингі болып есептелінеді. Топтың рейтінгі неғұрлым жоғары болған сайын оның алатын орны соғұрлым жоғары болады. Топ $K+1$ орынға ие болады, егер одан рейтингі \textbf{қатаң түрде үлкен} болатын дәл $K$ топ табылса. 

Барлық мүмкін топтасулардың ішінен Амантайдың тобының ең жоғары және ең төмен ала алатын орнын табыңыз. Амантай тізімде бірінші тұрған қатысушы. 

\InputFile
Бірінші қатарда бүтін $N$ саны беріледі. Келесі қатарда бос орындар арқылы $2 \cdot N$ бүтін сан беріледі. $1 \le a_i \le 10^5$, $1 \le i \le 2 \cdot N$.

\OutputFile
Ең жоғарғы және ең төменгі орын бола алатын екі санды шығарыңыз.

\Examples

\begin{example}
\exmp{3
999 3 1 2 1000 1
}{1 2
}%
\exmp{1
1540 1433
}{1 1
}%
\exmp{3
100000 100000 100000 100000 100000 100000
}{1 1
}%
\end{example}


Бірінші мысалда егер біз (999, 2) (3, 1) (1000, 1) қылып топтастырсақ, онда Амантайдың (999, 2) тобы  және (1000, 1) топ екеуі бірінші орынға ие болады, ал (3, 1) тобы үшінші орын алады. Ал егер де біз (999, 1) (1000, 2) (3, 1) қылып топтастырсақ онда Амантайдың тобы екінші орын алады. Барлық мүмкін болатын топтастырулардың ішінен, жоғарыда көрсетілген топтастырулар ең жоғарғы және ең төменгі орындарға сай келеді.

\Scoring
Берілген тапсырма төрт бөліктен тұрады:
\begin{enumerate}
\item $1 \le N \le 3$. Бағалануы $7$ ұпай.
\item $1 \le N \le 6$. Бағалануы $19$ ұпай.
\item $1 \le N \le 2500$. Бағалануы $31$ ұпай.
\item $1 \le N \le 10^5$. Бағалануы $43$ ұпай.
\end{enumerate}

Әр бөлік өзінен алдынғы бөліктер орындалғанда ғана бағаланады.

\end{problem}
