\begin{problem}{Контейнерлер мен қораптар}{A.in}{A.out}{0.5 секунд}{256 мегабайт}

Сiз Нурлаш и КО жүк көлiк компаниясының басты бағдарламалаушысысыз. Компанияға сорттайтын робот үшін жаңа функционал керек. Робот $N$ қорап үшiн жауап бередi, қораптар бiрден $N$-ға дейiн нөмiрленген. Робот екi түрлi операция орындай алады:

\begin{enumerate}
    \item $L$-дан $R$-ға дейін әр қораптың соңына нөмiрi $C$ деген контейнер қосу.
    \item $L$-дан $R$-ға дейiн әр қораптан соңғы контейнерді алып тастау.
\end{enumerate}

Сiзге робот орындаған операциялар берiлген. Әр қорап үшiн соңында қай нөмірдегі контейнер тұрғанын білу қажет.

\InputFile
Бiрiншi қатарда $N$ және $M$ ($1 \leq N, M \leq 10^5$) сандары берілген, қораптар және операциялардың саны. Одан кейінгі $M$ қатарда робот операциялары берiледi. Әр қатар үш саннан тұрады --- $L$, $R$, $C$ ($1 \leq L \leq R \leq 10^5$, $0 \leq C \leq 10^9$).

Егер $C = 0$ болса, ол екiншi операция белгiсi. Керiсiнше болса, бiрiншi.

Барлық сандар бүтiн. Операциялар арасында бос қораптан контейнер алып тастайтын операциялар жоқ.

 

\OutputFile
Бiр жолда $N$ сан шығарыныз. Бiрiншi ол бiрiншi қораптын соңғы контейнерінің нөмiрi, екiншi ол екiншi қораптын соңғы контейнерінің нөмiрi, тура солай ары қарай.

\Examples

\begin{example}
\exmp{5 3
1 5 1
2 4 0
4 5 10
}{1 0 0 10 10 }%
\end{example}


\Scoring
Берілген тапсырма екi бөліктен тұрады:
\begin{enumerate}
\item $1 \leq N, M \leq 1000$. Бағалануы $40$ ұпай.
\item $1 \leq N, M \leq 10^5$. Бағалануы $60$ ұпай.
\end{enumerate}

Әр бөлік өзінен алдынғы бөліктер орындалғанда ғана бағаланады.

\end{problem}
